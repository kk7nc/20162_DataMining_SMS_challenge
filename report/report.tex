\documentclass[a4paper,12pt]{report}
\usepackage[utf8]{vietnam}
\usepackage{amsmath}
\usepackage{amsfonts}
\usepackage{enumitem}
%\usepackage{amssymb}
\usepackage{graphicx}
%\usepackage{cases}
\usepackage{fancybox}
\usepackage{multirow}
\usepackage{longtable}
\usepackage{listings}
\usepackage[nottoc]{tocbibind}
\usepackage{indentfirst}
\usepackage[english]{babel}
\usepackage{float}
\PassOptionsToPackage{hyphens}{url}\usepackage{hyperref}  
\usepackage[left=3cm, right=2.00cm, top=2.00cm, bottom=2.00cm]{geometry}
%\lstset{
   %keywords={break,case,catch,continue,else,elseif,end,for,function,
   %   global,if,otherwise,persistent,return,switch,try,while},
%   language = Java,
%   basicstyle=\ttfamily \fontsize{12}{15}\selectfont,   
	% numbers=left,
%   frame=lrtb,
%tabsize=3
%}
\hypersetup{
    colorlinks,
    citecolor=black,
    filecolor=black,
    linkcolor=blue,
    urlcolor=red 
}
\setlength{\parskip}{0.6em}
\addto\captionsenglish{%
 \renewcommand\chaptername{Phần}
 \renewcommand{\contentsname}{Mục lục} 
 \renewcommand{\listtablename}{Danh sách bảng}
 \renewcommand{\listfigurename}{Danh sách hình vẽ}
 \renewcommand{\tablename}{Bảng}
 \renewcommand{\figurename}{Hình}
 \renewcommand{\bibname}{Tài liệu tham khảo}
}

\newtheorem{definition}{Định nghĩa}[chapter]
%\newtheorem{lema}{Bổ đề}[chapter]
%\newtheorem{theorem}{Định lý}[chapter]

\begin{document}
\thispagestyle{empty}
\thisfancypage{
\setlength{\fboxrule}{1pt}
\doublebox}{}

\begin{center}
{\fontsize{16}{19}\fontfamily{cmr}\selectfont TRƯỜNG ĐẠI HỌC BÁCH KHOA HÀ NỘI\\
VIỆN CÔNG NGHỆ THÔNG TIN VÀ TRUYỀN THÔNG}\\
\textbf{------------*******---------------}\\[1cm]
\includegraphics[scale=0.13]{hust.jpg}\\[1.3cm]
{\fontsize{32}{43}\fontfamily{cmr}\selectfont BÁO CÁO}\\[0.1cm]
{\fontsize{38}{45}\fontfamily{cmr}\fontseries{b}\selectfont MÔN HỌC}\\[0.2cm]
{\fontsize{20}{24}\fontfamily{phv}\selectfont Khai phá dữ liệu}\\[0.3cm]
{\fontsize{18}{20}\fontfamily{cmr}\selectfont \emph{Đề tài:  Vietnamese SMS Spam Filter Datamining Challenge }}\\[2cm]
\hspace{-5cm}\fontsize{14}{16}\fontfamily{cmr}\selectfont \textbf{Nhóm sinh viên thực hiện:}\\[0.1cm] 
\begin{longtable}{l c c}
Nguyễn Tuấn Đạt & 20130856 & CNTT2.02-K58 \\
Nguyễn Đức Mạnh & 20131518 & CNTT2.04-K58\\
Phan Anh Tú &   20134501 & CNTT2.01-K58\\
\end{longtable}

\hspace{-6cm}\fontsize{14}{16}\fontfamily{cmr}\selectfont \textbf{Giảng viên hướng dẫn:}\\[0.1cm]
\hspace{-2.7cm}\fontsize{14}{16}\fontfamily{cmr}\selectfont TS. Trịnh Anh Phúc \\[3.0cm]
\fontsize{16}{19}\fontfamily{cmr}\selectfont Hà Nội 5--2017
\end{center}
\newpage
\pdfbookmark{\contentsname}{toc}
\tableofcontents
%\listoftables
\listoffigures
\chapter{Tổng quan}
\section{Dữ liệu đầu vào}
Đầu vào của dữ liệu là tập \\
Trong challenge này, bọn em thử nghiệm qua nhiều phương pháp (SVM,Naive Bayes,KNN,Random forest, Word2Vec,...). Nhưng hai phương pháp cho hiệu quả nhất đó là Naive Bayes và K-nearest neighbors vì vậy trong báo cáo này bọn em sẽ trình bày 2 phương pháp trên.
\begin{itemize}
\item Phương pháp
\end{itemize}
\section{Chuẩn hóa vector đầu vào cho bài toán phân loại}
\chapter{Các phương pháp phân loại}
\section{Naive Bayes}
\subsection{Cơ sở lý thuyết}
\section{K-nearest neighbors}
\subsection{Cơ sở lý thuyết}
\chapter{Đánh giá}
\section{Naive Bayes}
\section{K-nearest neighbors}

\begin{thebibliography}{9}


\end{thebibliography}


\end{document}
