\documentclass[a4paper,12pt]{report}
\usepackage[utf8]{vietnam}
\usepackage{amsmath}
\usepackage{amsfonts}
\usepackage{enumitem}
%\usepackage{amssymb}
\usepackage{graphicx}
%\usepackage{cases}
\usepackage{fancybox}
\usepackage{multirow}
\usepackage{longtable}
\usepackage{listings}
\usepackage[nottoc]{tocbibind}
\usepackage{indentfirst}
\usepackage[english]{babel}
\usepackage{float}
\PassOptionsToPackage{hyphens}{url}\usepackage{hyperref}  
\usepackage[left=3cm, right=2.00cm, top=2.00cm, bottom=2.00cm]{geometry}
%\lstset{
   %keywords={break,case,catch,continue,else,elseif,end,for,function,
   %   global,if,otherwise,persistent,return,switch,try,while},
%   language = Java,
%   basicstyle=\ttfamily \fontsize{12}{15}\selectfont,   
	% numbers=left,
%   frame=lrtb,
%tabsize=3
%}
\hypersetup{
    colorlinks,
    citecolor=black,
    filecolor=black,
    linkcolor=blue,
    urlcolor=red 
}
\setlength{\parskip}{0.6em}
\addto\captionsenglish{%
 \renewcommand\chaptername{Phần}
 \renewcommand{\contentsname}{Mục lục} 
 \renewcommand{\listtablename}{Danh sách bảng}
 \renewcommand{\listfigurename}{Danh sách hình vẽ}
 \renewcommand{\tablename}{Bảng}
 \renewcommand{\figurename}{Hình}
 \renewcommand{\bibname}{Tài liệu tham khảo}
}

\newtheorem{definition}{Định nghĩa}[chapter]
%\newtheorem{lema}{Bổ đề}[chapter]
%\newtheorem{theorem}{Định lý}[chapter]

\begin{document}
\thispagestyle{empty}
\thisfancypage{
\setlength{\fboxrule}{1pt}
\doublebox}{}

\begin{center}
{\fontsize{16}{19}\fontfamily{cmr}\selectfont TRƯỜNG ĐẠI HỌC BÁCH KHOA HÀ NỘI\\
VIỆN CÔNG NGHỆ THÔNG TIN VÀ TRUYỀN THÔNG}\\
\textbf{------------*******---------------}\\[1cm]
\includegraphics[scale=0.13]{hust.jpg}\\[1.3cm]
{\fontsize{32}{43}\fontfamily{cmr}\selectfont BÁO CÁO}\\[0.1cm]
{\fontsize{38}{45}\fontfamily{cmr}\fontseries{b}\selectfont MÔN HỌC}\\[0.2cm]
{\fontsize{20}{24}\fontfamily{phv}\selectfont Khai phá dữ liệu}\\[0.3cm]
{\fontsize{18}{20}\fontfamily{cmr}\selectfont \emph{Đề tài:  Vietnamese SMS Spam Filter Datamining Challenge }}\\[2cm]
\hspace{-5cm}\fontsize{14}{16}\fontfamily{cmr}\selectfont \textbf{Nhóm sinh viên thực hiện:}\\[0.1cm] 
\begin{longtable}{l c c}
Nguyễn Tuấn Đạt & 20130856 & CNTT2.02-K58 \\
Nguyễn Đức Mạnh & 20131518 & \\
Phan Anh Tú &   20134501 & CNTT2.01-K58\\
\end{longtable}

\hspace{-6cm}\fontsize{14}{16}\fontfamily{cmr}\selectfont \textbf{Giảng viên hướng dẫn:}\\[0.1cm]
\hspace{-2.7cm}\fontsize{14}{16}\fontfamily{cmr}\selectfont TS. Trịnh Anh Phúc \\[3.0cm]
\fontsize{16}{19}\fontfamily{cmr}\selectfont Hà Nội 5--2017
\end{center}
\newpage
\pdfbookmark{\contentsname}{toc}
\tableofcontents
%\listoftables
\listoffigures

\chapter{Các phương pháp sử dụng}
Để có thể phân loại được các tin nhắn, trước hết cần chuyển đổi các tin nhắn dạng text (đầu vào của các phương pháp) về các vector số để máy tính có thể hiểu được; sau đó, chúng em đã tiến hành thử nghiệm nhiều phương pháp phân loại như: Navive-Bayes, k-nearest neighbors (KNN), support vector machine, ... Và chúng em nhận thấy kết quả trên 2 phương pháp Naive-Bayes, KNN là tốt hơn cả. Trong phần này chúng em sẽ trình bày về cách chuyển đổi vector 
\section{Naive-Bayes}

\section{K}

\chapter{Cài đặt}

\begin{thebibliography}{9}


\end{thebibliography}


\end{document}
