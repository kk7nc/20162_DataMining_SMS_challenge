\documentclass[a4paper,12pt]{report}
\usepackage[utf8]{vietnam}
\usepackage{amsmath}
\usepackage{amsfonts}
\usepackage{enumitem}
%\usepackage{amssymb}
\usepackage{graphicx}
%\usepackage{cases}
\usepackage{fancybox}
\usepackage{multirow}
\usepackage{longtable}
\usepackage{listings}
\usepackage[nottoc]{tocbibind}
\usepackage{indentfirst}
\usepackage[english]{babel}
\usepackage{float}
\PassOptionsToPackage{hyphens}{url}\usepackage{hyperref}  
\usepackage[left=3cm, right=2.00cm, top=2.00cm, bottom=2.00cm]{geometry}
%\lstset{
   %keywords={break,case,catch,continue,else,elseif,end,for,function,
   %   global,if,otherwise,persistent,return,switch,try,while},
%   language = Java,
%   basicstyle=\ttfamily \fontsize{12}{15}\selectfont,   
	% numbers=left,
%   frame=lrtb,
%tabsize=3
%}
\hypersetup{
    colorlinks,
    citecolor=black,
    filecolor=black,
    linkcolor=blue,
    urlcolor=red 
}
\setlength{\parskip}{0.6em}
\addto\captionsenglish{%
 \renewcommand\chaptername{Phần}
 \renewcommand{\contentsname}{Mục lục} 
 \renewcommand{\listtablename}{Danh sách bảng}
 \renewcommand{\listfigurename}{Danh sách hình vẽ}
 \renewcommand{\tablename}{Bảng}
 \renewcommand{\figurename}{Hình}
 \renewcommand{\bibname}{Tài liệu tham khảo}
}

\newtheorem{definition}{Định nghĩa}[chapter]
%\newtheorem{lema}{Bổ đề}[chapter]
%\newtheorem{theorem}{Định lý}[chapter]

\begin{document}
\thispagestyle{empty}
\thisfancypage{
\setlength{\fboxrule}{1pt}
\doublebox}{}

\begin{center}
{\fontsize{16}{19}\fontfamily{cmr}\selectfont TRƯỜNG ĐẠI HỌC BÁCH KHOA HÀ NỘI\\
VIỆN CÔNG NGHỆ THÔNG TIN VÀ TRUYỀN THÔNG}\\
\textbf{------------*******---------------}\\[1cm]
\includegraphics[scale=0.13]{hust.jpg}\\[1.3cm]
{\fontsize{32}{43}\fontfamily{cmr}\selectfont BÁO CÁO}\\[0.1cm]
{\fontsize{38}{45}\fontfamily{cmr}\fontseries{b}\selectfont MÔN HỌC}\\[0.2cm]
{\fontsize{20}{24}\fontfamily{phv}\selectfont Khai phá dữ liệu}\\[0.3cm]
{\fontsize{18}{20}\fontfamily{cmr}\selectfont \emph{Đề tài:  Vietnamese SMS Spam Filter Datamining Challenge }}\\[2cm]
\hspace{-5cm}\fontsize{14}{16}\fontfamily{cmr}\selectfont \textbf{Nhóm sinh viên thực hiện:}\\[0.1cm] 
\begin{longtable}{l c c}
Nguyễn Tuấn Đạt & 20130856 & CNTT2.02-K58 \\
Nguyễn Đức Mạnh & 20131518 & CNTT2.04-K58\\
Phan Anh Tú &   20134501 & CNTT2.01-K58\\
\end{longtable}

\hspace{-6cm}\fontsize{14}{16}\fontfamily{cmr}\selectfont \textbf{Giảng viên hướng dẫn:}\\[0.1cm]
\hspace{-2.7cm}\fontsize{14}{16}\fontfamily{cmr}\selectfont TS. Trịnh Anh Phúc \\[3.0cm]
\fontsize{16}{19}\fontfamily{cmr}\selectfont Hà Nội 5--2017
\end{center}
\newpage
\pdfbookmark{\contentsname}{toc}
\tableofcontents
%\listoftables
\listoffigures

\chapter{Tổng quan}
\section{Dữ liệu đầu vào}
Đầu vào của dữ liệu là tập \\
Trong challenge này, bọn em thử nghiệm qua nhiều phương pháp (SVM,Naive Bayes,KNN,Random forest, Word2Vec,...). Nhưng hai phương pháp cho hiệu quả nhất đó là Naive Bayes và K-nearest neighbors vì vậy trong báo cáo này bọn em sẽ trình bày 2 phương pháp trên.
\begin{itemize}
\item Phương pháp
\end{itemize}
\section{Chuẩn hóa vector đầu vào cho bài toán phân loại}
\chapter{Các phương pháp phân loại}
\section{Naive Bayes}
\subsection{Cơ sở lý thuyết}
Mục đích của các phương pháp học máy là học một hàm dựa trên dữ liệu cho trước (tập train) và dùng hàm đó để dự đoán dữ liệu tương lai (trong bài toán phân loại là gán nhãn lớp cho dữ liệu). Phương pháp Naive-Bayes giả thiết dữ liệu của bài toán cần phân loại được sinh ra theo một phân bố xác suất nào đó. 
\subsubsection{Các khái niệm cơ bản vễ xác suất}
Giả sử chugns ta có một sự kiện mà kết quả của nó mang tính ngẫu nhiên.
\begin{itemize}
\item 
\end{itemize}
\subsubsection{Định lý Bayes}
Cho hai biến ngẫu nhiên X, Y.Ta định nghĩa:
\begin{itemize}
\item Xác suất hợp của hai biến X, Y: $Pr(X=x, Y=Y)$ là xác suất để 
\end{itemize}

\section{K-nearest neighbors}
\subsection{Cơ sở lý thuyết}
\chapter{Đánh giá}
Vì tập dữ liệu khá nhỏ nên phương pháp bọn em sử dụng để đánh giá mô hình là : repeated hold-out. Chúng em thực hiện phương pháp hold out(80 train-20 test) 100 lần sau đó tính trung bình để đánh giá mô hình học máy.\\

Về lựa chọn tham số cho từng phương pháp học máy bọn em sử dụng phương pháp hold-out. Cụ thể sẽ được trình bày trong hai mục dưới đây
\section{Naive Bayes}
Với phương pháp naive-bayes bọn em thực hiện đánh giá trên 3 mô hình:
\begin{itemize}
\item Phương pháp naive bayes đơn thuần
\item Phương pháp naive bayes với tập từ điển các từ có xác suất harm spam cao build từ tập train.
\item Phương pháp naive bayes với tập train được thêm dữ liệu
\end{itemize}
\subsection{Tập từ điển harm spam}
Bọn thực hiện build một tập từ điển harm spam gồm 60 từ được lọc ra từ tập train. Nhưng từ trong từ điển là những từ thoản mãn 2 điều kiện sau:
Xét một từ x gọi CX là  tập câu chứa từ x
\begin{itemize}
\item $P_{CX}['Harm'] or P_{CX}['Spam'] >0.7$
\item $tf(x,d) >0.4$ tần số xuất hiện của từ đó trong 1 văn bản lớn hơn 40 \%
\end{itemize}
Kết quả cụ thể của ba mô hình thể hiện trong bảng sau:
\begin{longtable}{|c|c|}
\hline 
Mô hình & Độ chính xác \\ \hline
naive-bayes normal & 92.9\\ \hline
navie-bayes with dict & 92.1 \\ \hline
navie-bayes more data & 75.375\\ \hline
\end{longtable}


\section{K-nearest neighbors}
Với phương pháp KNN bọn em thực hiện chọn tham số mô hình (k,độ đo). 
\begin{longtable}{|c|c|c|c|c|c|}
\hline
Hàm khoảng cách & k=1 & k=3 & k=5 & k=7 & k=9\\
\hline
euclidean & 0.561 &0.495 &0.485&0.4955&0.485 \\
\hline 
jaccard & 0.9 & 0.9095 & 0.905 & 0.8995 & 0.879 \\ \hline
matching & 0.5655 & 0.4835 &  0.501 &0.4885 & 0.489 \\ \hline 
dice &0.8825 & 0.904 & 0.902 & 0.8955 & 0.8855 \\ \hline
kulsinski&  0.869 &0.8935& 0.878& 0.882 &0.875 \\ \hline 
rogerstanimoto&  0.5865& 0.483& 0.495 &0.4995& 0.487 \\ \hline
russellrao&  0.8495& 0.8695& 0.8505  &0.8505& 0.8505 \\ \hline 
sokalmichener& 0.5595& 0.505 &0.483 &0.492 &0.4755 \\ \hline 
sokalsneath& 0.896& 0.919  & 0.9075& 0.892 & 0.889 \\ \hline
\end{longtable}

Một vài định nghĩa sử dụng cho các hàm khoảng cách dưới đây:
\begin{itemize}
\item N số chiều của vector
\item NTT số chiều mà cả hai giá trị đều bằng 1
\item NTF số chiều mà giá trị của vector 1 bằng 1 vector 2 bằng 0
\item NFT số chiều mà giá trị của vector 2 băng 1 vector 1 bằng 0
\item NFF số chiều mà cả hai giá trị đều bằng 0
\item NNEQ NNEQ=NTF+NFT
\item NNZ NNZ=NTF+NFT+NTT 
\end{itemize}
Định nghĩa các hàm khoảng cách:
\begin{itemize}
\item euclidean : $\sqrt{\sum_1^N (x-y)^2}$
\item “jaccard”	JaccardDistance	NNEQ / NNZ
\item “matching”	MatchingDistance	NNEQ / N
\item “dice”	DiceDistance	NNEQ / (NTT + NNZ)
\item “kulsinski”	KulsinskiDistance	(NNEQ + N - NTT) / (NNEQ + N)\item “rogerstanimoto”	RogersTanimotoDistance	2 * NNEQ / (N + NNEQ)
\item “russellrao”	RussellRaoDistance	NNZ / N
\item “sokalmichener”	SokalMichenerDistance	2 * NNEQ / (N + NNEQ)
\item “sokalsneath”	SokalSneathDistance	NNEQ / (NNEQ + 0.5 * NTT)
\end{itemize}
\begin{thebibliography}{9}


\end{thebibliography}


\end{document}
